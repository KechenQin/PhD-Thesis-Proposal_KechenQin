%----------------------------------------------------------------------------------------
%	VALUES FOR THE THESIS
%----------------------------------------------------------------------------------------

\newcommand{\name}{\href{http://www.ccis.neu.edu/home/kechenqin/}{Kechen Qin}} % Author name
\newcommand{\thesistitle}{Solve Real World Tasks using Latent Information} % Title of the thesis
\newcommand{\submissiondate}{April, 2020} % Submission date "Month, year"
\newcommand{\supervisor}{\href{https://www.khoury.northeastern.edu/people/jay-javed-aslam/}{Javed A. Aslam}} % Supervisor name
\newcommand{\committee}{\href{https://www.khoury.northeastern.edu/people/jay-javed-aslam/}{Javed A. Aslam}} % Co-Supervisor name, comment this line if there is none
\newcommand{\committeemore}{\href{https://www.khoury.northeastern.edu/people/virgil-pavlu/}{Virgil Pavlu}}



%----------------------------------------------------------------------------------------
%	BIBLIOGRAPHY STYLE (pick the style you want)
%----------------------------------------------------------------------------------------

\usepackage[square, numbers, sort&compress]{natbib} % for bibliography - Square brackets, citing references with numbers, citations sorted by appearance in the text and compressed (as in [4-7])
%\usepackage[longnamesfirst,round]{natbib} % Natural Sciences bibliography

\bibliographystyle{Preamble/physics_bibstyle} % You may use a different style adapted to your field
%\bibliographystyle{unsrtnat} % You may use a different style adapted to your field


%----------------------------------------------------------------------------------------
%	YOUR PACKAGES (be careful of package interaction)
%----------------------------------------------------------------------------------------

\usepackage{amsthm,amsmath,amssymb,amsfonts,bbm}% Math symbols
\usepackage{amsbsy,bm}
\usepackage{float}
\usepackage{times}
\usepackage{latexsym}
\usepackage{csvsimple}
\usepackage{url}
\usepackage{graphicx}  %Required
\usepackage{subfigure}
\usepackage{multirow}
\usepackage{xcolor}
\usepackage{algorithm}
\usepackage[noend]{algorithmic}
\usepackage{pdfpages}
\usepackage{hyperref}

% \usepackage[numbers]{natbib}
% \usepackage{ftnxtra}
% \usepackage{fnpos}
% \usepackage{subfig}

% other packages
% \usepackage{fullpage}
% \usepackage{amsmath}
% \usepackage{cite}
% \usepackage{tabularx}
% \usepackage{multirow}
% \usepackage{graphicx}
% \usepackage{amsfonts}
% \usepackage{wrapfig}
% % \usepackage{algpseudocode}

% \usepackage{caption}
% \usepackage{hyperref}       % hyperlinks
% \usepackage{url}            % simple URL typesetting
% \usepackage{booktabs}       % professional-quality tables
% \usepackage{amsfonts}       % blackboard math symbols
% \usepackage{nicefrac}       % compact symbols for 1/2, etc.
% \usepackage{microtype}      % microtypography
% \usepackage{bm}
% \usepackage{bbm}
% \usepackage{amsmath}
% \usepackage{color} 
% \usepackage[dvipsnames]{xcolor}
% \usepackage{geometry}
% \usepackage{pdfpages}

% %----------------------------------------------------------------------------------------
%	YOUR DEFINITIONS AND COMMANDS
%----------------------------------------------------------------------------------------

% New Commands
\newcommand{\bea}{\begin{eqnarray}} % Shortcut for equation arrays
\newcommand{\eea}{\end{eqnarray}}
\newcommand{\e}[1]{\times 10^{#1}}  % Powers of 10 notation
\newcommand{\todo}[1]{\textcolor{red}{[TODO]: {#1}}}
\DeclareMathOperator*{\argmax}{arg\,max}
\DeclareMathOperator*{\argmin}{arg\,min}

% Defining a theorem box for Criteria
\newtheorem{critere}{Criterion}
\newcommand{\crit}[2]{
\begin{center}  
\fbox{ \begin{minipage}[c]{0.9 \textwidth}
\begin{critere}
\textbf{\textup{ #1}} --- #2
\end{critere}
\end{minipage}  } \end{center}
}